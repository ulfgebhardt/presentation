\section{Gems}
\frame{
	\begin{block}{}
  	\begin{center}
  	\huge{System - Gems}
	\end{center}
  	\end{block}
}			

\subsection{API}
\frame{
	\frametitle{Anwendungsbereich von System}
	System kann in PHP-basierten Anwendungen eingesetzt werden.
	\begin{block}{Anwendungsbereich}
	\begin{itemize}
	\item{Websites}
   	\item{Webtools}
	\item{Webapps}
	\end{itemize}
	\end{block}
}

\subsection{Quick Query}
\frame{
	\frametitle{Features von System}
	System vereichfacht die Entwicklung von PHP basierten Anwendungen
	\begin{block}{Features}
	\begin{itemize}
	\item{Kapselung}
   	\item{REST Schnittstelle}
	\item{Moderne Webtechnologien}
	\item{Utilities}
	\item{Modulare GUI für administrative Aufgaben}
	\end{itemize}
	\end{block}
	Teilintegration möglich
}
	
\subsection{Kapselung - Eine Seite}
\frame{
	\frametitle{Klassische Struktur von PHP Projekten}
	\begin{block}{}
	Die klassische Strucktur von PHP Projekten orientiert sich oft an der HTML Struktur.
	In den entsprechenden HTML-Div's werden weitere PHP-Scripts included. Das birgt Nachteile.
	\begin{itemize}
	\item{HTML Code ist chaotisch}
	\item{Programm ist eine Datei, zerteilt in Abschnitte}
	\item{Definitionen in anderen Abschnitten des Programms}
	\item{Spezialwissen notwendig für die Wartung}
	\end{itemize}
	\end{block}
}

\subsection{?}
\frame{
	\frametitle{Klassische Struktur von PHP Projekten}
	\begin{block}{}
	Die klassische Strucktur von PHP Projekten orientiert sich oft an der HTML Struktur.
	In den entsprechenden HTML-Div's werden weitere PHP-Scripts included. Das birgt Nachteile.
	\begin{itemize}
	\item{HTML Code ist chaotisch}
	\item{Programm ist eine Datei, zerteilt in Abschnitte}
	\item{Definitionen in anderen Abschnitten des Programms}
	\item{Spezialwissen notwendig für die Wartung}
	\end{itemize}
	\end{block}
}